%% Key Stage 3 Computing
%% Tim Eaglestone, 2013
%% Creative Commons Licence
%% Key Stage 3 Computing Scheme of Work by Tim Eaglestone is licensed under a Creative Commons Attribution-NonCommercial 3.0 Unported License.
%% Based on a work at https://github.com/eaglestone/ks3-computing. 
%%%%%%%%%%%%%%%%%%%%%%%%%%%%%%%%%%%%%%%%%%%%%%%%%%%%%%%%%%%%%

%----------------------------------------------------------
%
\documentclass[a4paper,10p]{article}
%
%----------------------------------------------------------
%
\usepackage{amsmath}%
\usepackage{amsfonts}%
\usepackage{amssymb}%
\usepackage{graphicx}


\begin{document}

\title{Developing a new computing scheme of work for Key Stage 3}
\author{Tim Eaglestone}
\date{April, 2013}
\maketitle

\section{Introduction}
\textit{Note: This scheme of work is a draft in response to the publication of the consultation National Curriculum for maintained schools in England issued in February 2013. As such it is a work in progress and likely to change. It is also incomplete: my intention is to iteratively develop units having tested ideas with my classes, consulted colleagues and watched the ongoing debate develop.}

\textit{Due to the scale of the change from ICT to Computing, and the need to reconcile some competing demands from both disciplines, I will start by giving a rationale in more detail than I would normally prefer.}

\section{Aims}
The aim of this scheme of work is to provide a current, engaging and coherent experience of computing for students in their first three years of secondary school. By the end of the key stage, students should have a sound understanding of how digital technologies work, how they interconnect and inter-operate, and the social and individual implications of living in a networked world. Students should also gain hands on experience in creating software and information systems. Furthermore, the computational thinking element should provide students with the conceptual tools needed to intelligently exploit digital technologies to solve problems.

This scheme of work should also afford progression for the majority of students to key stage 4 study in computing or information and communication technology at GCSE both in regard the breadth of content and its difficulty.

\section{Syllabus Content}
\label{sec:SyllabusContent}


The syllabus is composed of three main elements: computational thinking; computers and networks (for want of a better word or phrase), and digital literacy (again for want of a better phrase).

\subsection{Computational thinking}
\label{sec:ComputationalThinking}
This has four aspects:

\begin{description}
    \item[Decomposition:] breaking problems and tasks down into discrete, manageable steps
    \item[Patterns:] Recognising patterns, common features, similarities and differences as a basis for problem solving.
    \item[Abstraction:] Representing ideas, processes, relationships and data in general terms. This also entails deciding what not to include when developing models or solutions..
    \item[Algorithms:] Producing a step-by-step strategy to solve a problem.
\end{description}

\subsection{Computers and networks}
\label{sec:ComputersAndNetworks}


From the draft Programme of Study for Computing (Feb 2013), this section covers:
\begin{description}
\item[Computer systems:] Understanding the hardware and software components that make up networked computer systems, how they interact, and how they affect cost and performance. How instructions are stored and executed.
    \item[Networks and the Internet:] Explaining how networks such as the Internet work and their characteristics. Analyse how networks affect the user and understand issues of e-safety.
    \item[Control:] Understanding how computers can monitor and control physical systems.
\end{description}

\subsection{Digital Literacy}
\label{sec:DigitalLiteracy}

Again from the draft Programme of Study this involves:
\begin{description}
    \item[Project management: Undertaking creative projects that involve selecting, using, and combining multiple applications, preferably across a range of devices, to achieve challenging goals, including collecting and analysing data and meeting the needs of known users.
    \item[IT Skills: Being able to create, reuse, revise and repurpose digital information and content with attention to design, intellectual property and audience.
\end{description}
In addition to the draft Programme of Study:
\begin{description}
    \item[Information literacy: Understanding validity and bias on the Internet. Explaining the difference between data, information and knowledge. Managing personal information and your digital footprint.
\end{description}
The content described above satisfies, and exceeds, the minimum requirement set out in the draft Programme of Study for Computing (Feb, 2013).

\section{Curriculum Structure}
\label{sec:CurriculumStructure}

In line with current research from UCLA, the course is structured in a way that facilities the interleaving of key concepts as they cut across several units of work. Hence, the scheme of work can be read as a series of units that give students the opportunity to experience a range of different contexts, packages, computer languages, projects and so on. But cutting across these units are key concepts such as decomposition and abstraction that are repeatedly met in differing contexts (the units) in order to support the concept of spacing See Bjork.

This approach leads us to separate the elements described above into content for the units and underlying concepts to thread throughout. The 10 elements describes above form headings for Yearly Teaching Objectives. Covering the objectives will ensure that the breadth of the curriculum is covered at the appropriate level for each year group. The Yearly Teaching Objectives will also ensure coherence and progression from years 7 through to 9.


\section{Pitching the course}
\label{sec:PitchingTheCourse}

With the disapplication of the ICT programme and study and publication of the draft Computing Programme of Study, we have the opportunity to pitch our scheme of work and lessons to our local needs.

Most of the students in my school typically arrive in year 7 at roughly level 4 in ICT judged by the old attainment targets. We are currently aiming for progression to the OCR Computing GCSE and the Cambridge iGCSE in IT. I will use the OCR course and the old APP as a benchmark for progression from years 7 to 9 in the absence of any additional guidance or statutory requirements.

We offer the OCR Computing GCSE as a two year course in years 10 and 11. From the OCR Spec (July 2012):
Grade F


\begin{quotation}
	Candidates recall, select and communicate a basic knowledge and understanding computer hardware, software and other related technologies.
			
	They identify, with guidance, the information relevant to solve them from the context of the problem.
	
	They apply limited knowledge, understanding and skills to design and implement basic computer programs which solve these problems. In their solutions they use simple models, collect some necessary data, use simple instructions to process the data and present the results.
	
	They sometimes review their work and provide comments on how they and others use computer technology to solve problems and make simple modifications to improve their work.
	
	Candidates demonstrate some awareness of the need for safe, secure and responsible practices.
	
	They use ICT to communicate, demonstrating limited awareness of purpose and audience.
\end{quotation}


Grade C


\begin{quotation}
	Candidates recall, select and communicate a good knowledge and understanding of the function, application merits and implications of a range of computer hardware, software and other related technologies.
	
	They analyse problems, identifying and collecting some information relevant to solve them from the context of the problem. They apply knowledge, understanding and skills to design and implement computer programs which solve these problems. In their solutions, they model situations, acquire a input data, sequence instructions, manipulate and process data and present the results of the processing in a mostly appropriate format.
	
	They review their work and evaluate the way they and others use computer technology to solve problems and make improvements on their work where appropriate.
	
	Candidates work using safe, secure and responsible practices. They use ICT to communicate, demonstrating consideration of purpose and audience.
\end{quotation}


Grade A


\begin{quotation}
	Candidates recall, select and communicate a thorough knowledge and understanding of the function, application, merits and implications of a broad range of computer hardware, software and other related technologies.
	
	They systematically analyse problems, identifying and collecting the information required to solve them from the context of the problem. They apply knowledge, understanding and skills to design and implement effective computer programs which solve these problems. In their solutions, they effectively model situations, acquire and validate input data, sequence instructions, manipulate and process data and present the results of the processing in an appropriate format.
	
	They work systematically and critically evaluate the way they and others use computer technology to solve problems. They iteratively review their work and make improvements where appropriate.
	
	Candidates work systematically and understand and adopt safe, secure and responsible practices.They use ICT to communicate effectively, demonstrating a clear sense of purpose and audience.
\end{quotation}


Comparing the above statements to the previous National Curriculum levels and, specifically, the Assessing Pupil Progress guidance, it appears as though the Grade C descriptors align closest with level 7; Grade F looks as though it aligns to the level 4 to 5 crossover; and Grade A is Exceptional Performance.

Hence, I will benchmark the Yearly Teaching Objectives for Year 7 against the Grade F criteria and aspire to the Grade C criteria by the end of Year 9.
Assessment

At the time of writing, we are still waiting for guidance and clarification on assessment and measures of progression. This section is likely to change in due course.

\section{Ongoing assessment}
\label{sec:OngoingAssessment}

The each unit has a column marked �performance� that lists a set of activities that might be useful when inferring learning. As such, they could be used as the basis of assessment for learning activities or sources of immediate feedback when planning individual lessons.

\subsection{End of unit assessment}
\label{sec:EndOfUnitAssessment}

Assessment at the end of each unit will have the joint aims of identifying what students have remembered and their ability to apply their learning.

\section{Sources}
\label{sec:Sources}

I am grateful to the following people and bodies who have published material freely that has informed my thinking. This is not an exhaustive list and it will change.

The Learning Spy: David Didau: A big inspiration as a teacher. Amongst other things, he put me on to Bjork through his blog and Tweeting.

Computing at School: The Computing At School Working Group (CAS) is a grass roots organisation that aims to promote the teaching of Computing at school. CAS is a collaborative partner with the BCS through the BCS Academy of Computing, and has formal support from other industry partners.

Google- Exploring Computational Thinking: Google have been active in engaging with schools for a while. Much if their material is based from work in the US. I have used their broad headings for computational thinking as the basis of the computational thinking elements of this curriculum.

CS Priciples- National Science Foundation: Computer Science: Principles is a new course under development that seeks to broaden participation in computing and computer science. Development is being led by a team of computer science educators organized by the College Board and the National Science Foundation. Pilots are ongoing at the high school and college levels.

The Beauty and Joy of Computing is an introductory computer science curriculum developed at the University of California, Berkeley, intended for non-CS majors at the high school junior through undergraduate freshman level.

Draft Australian Curriculum: Technologies February 2013 (accessed 21.02.1)

Creative Commons Licence
Key Stage 3 Computing Scheme of Work by Tim Eaglestone is licensed under a Creative Commons Attribution-NonCommercial 3.0 Unported License.
Based on a work at https://github.com/eaglestone/ks3-computing. 






\end{document}
